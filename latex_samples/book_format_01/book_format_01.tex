\documentclass[a4paper,12pt]{book}
% - \documentclass[a4paper,12pt,landscape,twocolumn]{book}
\usepackage[english]{babel}
\usepackage{blindtext}
\usepackage[a4paper, inner=1.5cm, outer=3cm, top=2.5cm, bottom=2.5cm, bindingoffset=1.5cm]{geometry}
%- For line spacing
\usepackage[onehalfspacing]{setspace}
%-- For math symbols
\usepackage{amsmath}


%--- Preamble : here you set new rules
    %--- For foonote  (else you can use \protect\footnote{Your note here} )
\renewcommand{\footnoterule}{\noindent\smash{\rule[3pt]{\textwidth}{0.4pt}}}
    %--- You can use this to name the contents, real place is tableofcontents below
\renewcommand{\contentsname}{Table of Contents}
    %--- Theorem and definitions
\newtheorem{thm}{Theorem}
\newtheorem{dfn}[thm]{Definition}





%--- Customizing headers: Start
\usepackage{fancyhdr}
\fancyhf{}
\fancyhead[LE]{\leftmark}
\fancyhead[RO]{\nouppercase{\rightmark}}
\fancyfoot[LE,RO]{\thepage}
    %Switches to the page style fancy from this point onwards
\pagestyle{fancy}
%--- Customizing headers: End


%--- From here the document begins
\begin{document}

%-- add table of contents
\tableofcontents


%----- Optional - this breaks book into parts
\part{Part One's Name}


\chapter[Chap1Name]{Exploring the page layout}
\begin{flushright} In this chapter we will study the layout of pages. \end{flushright}
\section[SectionName]{Some filler text}
\blindtext
\section[section name]{A lot more filler text}
 %-- Footnote example
More dummy text will follow. A lot more filler text \protect\footnote{to fill thepage}

More dummy text will follow. Another example hre \protect\footnote{for note rule}

footnoterule

\subsection[subsection name]{Plenty of filler text}
\blindtext[10]

%----- Optional - this breaks book into parts
\part{Part Two's Name}

\chapter[Chap2Name]{Exploring the page layout}
\begin{flushright} In this chapter we will study the layout of pages. \end{flushright}
\section[SectionName]{Some filler text}
\blindtext
\section[section name]{A lot more filler text}
More dummy text will follow.
\subsection[subsection name]{Plenty of filler text}
\blindtext[3]


The quadratic equation
\begin{equation}\label{quad}
 ax^2 + bx + c = 0,
\end{equation}
where \( a, b \) and \( c \) are constants and \( a \neq 0 \),has two solutions for the variable \( x \):
\begin{equation}\label{root}
 x_{1,2} = \frac{-b \pm \sqrt{b^2-4ac}}{2a}.
\end{equation}
%-- To get an unnumbered displayed equation, we surround the formula with
If the \emph{discrimimant} \( \Delta \) with
\[
 \Delta = b^2 - 4ac
\]
is zero, then the equation (\ref{quad}) has a double solution:
(\ref{root}) becomes
\[
 x = - \frac{b}{2a}.
\]

 %--- You can also use "$$" but the braces \[... \] manage vertical spacing better
$$
 \Delta = b^2 - 4ac
$$


%---- Embedding math expresssions in text
This is how we embedd math expressions in text \begin{math} (x+y)^2 = x^2 + y^2 + 2xy \end{math} . Even writing again it in a different way split across cleanly across lines
\begin{math}
  (x+y)^2 = x^2 + y^2 + 2xy
\end{math}
like this(I prefer this for clarity). Also you may use this way
\(
  (x+y)^2 = x^2 + y^2 + 2xy
\)
 . It works as well.


%----- To display centered formula
If you want centered formula, use this way
\begin{displaymath}
  (x+y)^2 = x^2 + y^2 + 2xy
\end{displaymath} See how it looks!  It creates some spacing as well(make sure to compare with other examples above in terms of spacing.). Same can be done as shortcut
\[
  (x+y)^2 = x^2 + y^2 + 2xy
\]


%----- Numbering equations
Some examples on Numbering equations:
\begin{equation} \label{key}
 (x+y)^2 = x^2 + y^2 + 2xy
\end{equation}

Use the multline environment to span a long equation over three lines
\begin{multline}
\sum = a + b + c + d + e \\
        + f + g + h + i + j \\
        + k + l + m + n
\end{multline}


Now we handle a system of equations. Use the gather environment to add these equations. Again, end lines with
\begin{gather}
 x + y + z = 0 \\
 y - z = 1
\end{gather}


Commonly, equation systems are aligned at the equal sign. Let's do this. Use the $\&$ symbol to mark the point that we wish to align:
\begin{align}
 x + y + z &= 0 \\
 y - z &= 1
\end{align}

Similary use \emph{flalign} and \emph{alignat} can be used. Try them

Inserting text into formulas, here we try this
\begin{equation} \label{key2}
 (x+y)^2 = \text{first term: } x^2 + \text{second term:} y^2 +  \text{third term: } 2xy
\end{equation}

Some sample operators:
\(
  \lim_{n=1, 2, \ldots} a_n, \qquad \max_{x<X} x
\)

Building math structures, e.g arrays
\[
 A = \left(
 \begin{array}{cc}
 a_{11} & a_{12} \\
 a_{21} & a_{22}
 \end{array}
 \right)
\]

Writing binomial coefficients:
\[
  \binom{n}{k} = \frac{n!}{k!(n-k)!}
\]

Typesetting matrices
\[
A = \begin{pmatrix}
      a_{11} & a_{12} \\
      a_{21} & a_{22}
    \end{pmatrix}
\]
You may notice that the parentheses are closer to the matrix entries than in the array example.

Got a cool feature, underbaces and overbraces:
\[
N = \underbrace{1 + 1 + \cdots + 1}_n
\]

%---- Writing theorems and definitions
\begin{dfn}
A quadratic equation is an equation of the form
  \begin{equation} \label{quad}
    ax^2 + bx + c = 0,
  \end{equation}
  where \( a, b \) and \( c \) are constants and \( a \neq 0 \).
\end{dfn}

\begin{thm}
A quadratic equation (\ref{quad}) has two solutions for the variable
\( x \):
\begin{equation}
 \label{root}
 x_{1,2} = \frac{-b \pm \sqrt{b^2-4ac}}{2a}.
\end{equation}
\end{thm}





%---- Appendix Start
\appendix
\cleardoublepage
\addtocontents{toc}{\bigskip}
\addcontentsline{toc}{part}{Appendix}
\chapter{Glossary}
\chapter{Symbols}
%---- Appendix End




\end{document}
