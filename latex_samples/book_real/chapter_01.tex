
\chapter[]{ Polynomials And Equations}
\epigraph{Politics is for the present, while our equations are for the eternity.}{- Albert Einstein}

\section[Problem Set-01]{Problem Set-01 \begin{flushright}1 Hour\end{flushright}}


\begin{problem}
Find the number of postive roots of the equation $ x^{11} - 3 x^5 + x^4 + 1 = 0$\begin{flushright}[4] marks\end{flushright}
\end{problem}

\begin{problem}
For which of the following values of $ m $ , the roots of the equation $ 2x^2 - x - 1 = 0$ lie inside the roots of the equation $x^2 + (2m - m^2)x - 2m^3 = 0$ \
\begin{enumerate} [label=(\alph*)]
\item $ m \in (\frac{1}{2}, 1]$
\item $ m \in (\frac{1}{3}, 1)$
\item $ m \in (\frac{1}{4}, 1)$
\item $ m \in (\frac{1}{4}, \infty)$
\end{enumerate}
\begin{flushright}[6, -1] marks\end{flushright}
\end{problem}

\begin{problem}
Number of integral values of $x$ for which the equation
$(x+3 - 4(x-1)^{\frac{1}{2}})^{\frac{1}{2}} - (x+8 - 6(x-1)^{\frac{1}{2}})^{\frac{1}{2}} = 1$ is satified if $\cdots$ \

\begin{flushright}[6] marks\end{flushright}
\end{problem}

\section[Answer Set-01]{Answer Set-01}
% -----------  answers
\begin{answer}
1
\end{answer}
\begin{answer}
c
\end{answer}
\begin{answer}
6
\end{answer}

\newpage

\section[Solution Set-01]{Solution Set-01}
%---------------- Solutions
	\begin{solution}
		By A.M. $\ge$ G.M  (true for all $+ve$ Real numbers).

		$\frac{x^{11} + x^4 + 1}{3} \ge \sqrt[3]{x^{11} . x^4 . 1} = x^5$  $(\because x \ge 0)$

		$\Rightarrow \frac{x^11 + x^4 + 1}{3} \ge 3x^5$

		equality only when $x^{11} = x^4 = 1$  $\Rightarrow$ $x=1$

		$\therefore $ only $x=1$ is the positive real number satisfying the equation.
	\end{solution}

	\begin{solution}
		$x^2 + (2m - m^2)x - 2m^3 = 0 \Rightarrow x = -\frac{1}{2}, 1$
    Consider the equation $x^2 + (2m - m^2)x - 2m^3 = 0$    $f(x) = 0$

		For real and distinct roots $D > 0  \Rightarrow$
		     \begin{align*}
					(2m - m^2)^2+8m^3 &> 0  \tag{1}
				 \end{align*}

	  $-\frac{1}{2}$ and $1$ lie inside the roots of $x^2 + (2m - m^2)x - 2m^3 = 0$

	\begin{center}
	 \begin{tikzpicture}
		 \begin{axis}[
		 % width and height if axis, adjust to your liking
		 width=12cm,
		 height=6cm,
		 xtick=\empty, % remove all ticks from x-axis
		 ytick=\empty, % ditto for y-axis
		 xlabel=$x$,
		 ylabel=$y$,
		 axis lines=center, % default is to make a box around the axis
		 domain=-2.5:2.5,
		 samples=100]
	 \addplot [red] {x^2-x-2};
	 \filldraw (-0.5,0) circle[radius=2.0pt];
	 \filldraw (1.0,0) circle[radius=2.0pt];
	 \addplot[mark=*] coordinates {(-0.5,0)} node[pin=90:{$-\frac{1}{2}$}]{} ;
	 \addplot[mark=*] coordinates {(1,0)} node[pin=90:{$1$}]{} ;
		\end{axis}

	 \end{tikzpicture}
 \end{center}

    $\Rightarrow f \left( -\frac{1}{2} \right) < 0 $ and $ f(1) < 0 $

    \begin{flalign*}
	    f \left( -\frac{1}{2} \right) < 0 & \Rightarrow m > \frac{1}{4}  \tag{2} \\
	           f(1) < 0 & \Rightarrow  m \in \left( -\infty, -1 \right) \cup \left( -\frac{1}{2}, 1 \right) \tag{3}
    \end{flalign*}

		$\therefore $ The value of $m$ satisfying both the above conditions: $ m \in \left( \frac{1}{4}, 1 \right)$
 \end{solution}


 %--- Next solutions
 \begin{solution}
	 Put $(x-1)^{\frac{1}{2}} = y$   $\Rightarrow x = y^2 + 1$
	 $\therefore$ the equation becomes
	 \begin{flalign*}
		 (y^2 + 4 - 4y)^{\frac{1}{2}} + (y^2 + 9 - 6y)^{\frac{1}{2}} &= 1 \\
		 [(y-2)^2]^{\frac{1}{2}} + [(y-3)^2]^{\frac{1}{2}}  &= 1 \\
		 \Rightarrow |y-2| + |y -3| &= 1
	 \end{flalign*}
   Above is satisfied for all $y \in [2,3]$
	  \begin{equation*}
			\begin{alignedat}{2}
			\therefore\quad&&                 2 &\le y \le 3 \\
		  \therefore\quad&&                 5 &\le y^2 + 1 \le 10
		\end{alignedat}
	\end{equation*}
	$\Rightarrow   5 \le x \le 10 $   $\therefore$  5, 6, 7, 8, 9, 10 are the integer values of $x$ satisfying the original equation.


		%$\Righarrow 5 \le x \le 10 $
 \end{solution}
