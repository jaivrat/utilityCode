\usepackage[dvipsnames]{xcolor}
\usepackage[many]{tcolorbox}

\usepackage[english]{babel}
\usepackage[utf8]{inputenc}
\usepackage[T1]{fontenc}
\usepackage{lmodern}
\usepackage{microtype}
\usepackage{natbib}
\usepackage{tocbibind}
\usepackage{amsmath}
\usepackage{amsthm}
\usepackage{blindtext}
% \usepackage[a5paper]{geometry}
\usepackage[a4paper, inner=1.5cm, outer=3cm, top=2.5cm, bottom=2.5cm, bindingoffset=1.5cm]{geometry}
%- For line spacing
\usepackage[onehalfspacing]{setspace}
%-- For math symbols
\usepackage{amsmath}
\usepackage{tikz,pgfplots}

%-- This introduces hyper rereferences in document : The print version
%-- would not be affected—neither the bookmarks nor those red frames would be printed
\usepackage{hyperref}

%-- Add this
\usepackage{titlesec}

%-- for drawings
\usepackage{tikz}

%--- Preamble : here you set new rules
    %--- For foonote  (else you can use \protect\footnote{Your note here} )
\renewcommand{\footnoterule}{\noindent\smash{\rule[3pt]{\textwidth}{0.4pt}}}
    %--- You can use this to name the contents, real place is tableofcontents below
\renewcommand{\contentsname}{Table of Contents}

%--- Customizing headers: Start
\usepackage{fancyhdr}
\fancyhf{}
\fancyhead[LE]{\leftmark}
\fancyhead[RO]{\nouppercase{\rightmark}}
\fancyfoot[LE,RO]{\thepage}
    %Switches to the page style fancy from this point onwards
\pagestyle{fancy}
%--- Customizing headers: End


%-- to specify layout and font of the chapter headings
%-- titlesec, which provides a comprehensive interface for customizing headings, of parts,
%-- chapters, sections, and even smaller sectioning parts down
%-- to subparagraphs.

\titleformat{\chapter}[display]
 {\normalfont\sffamily\Large\bfseries\raggedright}
 {\chaptertitlename\ \thechapter}{0pt}{\Huge}

%-- define the section heading by calling \titleformat again
\titleformat{\section}
 {\normalfont\sffamily\large\bfseries\raggedright}
 {\thesection}{1em}{}

%--  to adjust the chapter headings spacing:
\titlespacing*{\chapter}{0pt}{30pt}{20pt}


\newtheorem{thm}{Theorem}[chapter]
\newtheorem{lem}[thm]{Lemma}
\theoremstyle{definition}
\newtheorem{dfn}[thm]{Definition}

%--- pdf medatadate:
\hypersetup{pdfauthor={Chimpoo},
 pdftitle={The Big Book of Equations},
 pdfsubject={Solving Equations and Equation Systems},
 pdfkeywords={equations,mathematics}}


%--- Those quotations
\usepackage{epigraph}
\setlength{\epigraphwidth}{0.35\linewidth}
\setlength{\epigraphrule}{0pt}
\renewcommand*{\textflush}{flushright}
\renewcommand*{\epigraphsize}{\normalsize\itshape}






%-----------------------------------------------------------------------
%---------------------FROM THAT EXAMPLE NOTEX------------------------------
%-----------------------------------------------------------------------




% ----------------------------------------------------------------------
%           User Top Level Packages: Required
% ----------------------------------------------------------------------

\usepackage{marginnote,sidenotes,fancyhdr,titlesec,geometry}
\usepackage[dvipsnames]{xcolor}
\usepackage[many]{tcolorbox}

% ----------------------------------------------------------------------
%           User Top Level Packages: Additional & Styling
% ----------------------------------------------------------------------

\usepackage[T1]{fontenc}                            % Font Styling
\usepackage{lmodern,mathrsfs}

\usepackage[shortlabels]{enumitem}

\usepackage{mathtools,amssymb,amsfonts,amsthm,bm}   % Math Presets
\usepackage{array,tabularx,booktabs}                % Table Presets
\usepackage{graphicx,wrapfig,float,caption}         % Figure Presets
\usepackage{setspace,multicol}                      % Text Presets
\usepackage{tikz,physics,cancel}                    % Physics Presets




% ----------------------------------------------------------------------
%           User Created Environments
% ----------------------------------------------------------------------


\newtheoremstyle{mystyle}%
  {}%
  {}%
  {}%
  {}%
  {\sffamily\bfseries}%
  {.}%
  { }%
  {}%

\renewenvironment{proof}{{\sffamily\bfseries Proof. }}{\qed}

\theoremstyle{mystyle}{
  \newtheorem*{remark}{Remark}
}

\theoremstyle{mystyle}{
  \newtheorem{definition}{Definition}[section]
}

\theoremstyle{mystyle}{
  \newtheorem{theorem}{Theorem}[section]
}

\theoremstyle{mystyle}{
  \newtheorem{lemma}[theorem]{Lemma}
}

\theoremstyle{mystyle}{
  \newtheorem*{example}{Example}
}

\theoremstyle{definition}{
    \newtheorem*{exercise}{Exercise}
}

% ----------------------------------------------------------------------
% start: By jv
\theoremstyle{mystyle}{
  \newtheorem{problem}{Problem}[section]
}
\theoremstyle{mystyle}{
  \newtheorem{answer}{Answers}[section]
}
\theoremstyle{mystyle}{
  \newtheorem{solution}{Solution}[section]
}
% end: By jv
% ----------------------------------------------------------------------



\tcolorboxenvironment{definition}{
  boxrule=0pt,
  boxsep=2pt,
  colback={White!90!Cerulean},
  enhanced jigsaw,
  borderline west={2pt}{0pt}{Cerulean},
  sharp corners,
  before skip=10pt,
  after skip=10pt,
  breakable,
}

\tcolorboxenvironment{theorem}{
  boxrule=0pt,
  boxsep=2pt,
  colback={White!90!Dandelion},
  enhanced jigsaw,
  borderline west={2pt}{0pt}{Dandelion},
  sharp corners,
  before skip=10pt,
  after skip=10pt,
  breakable,
}

\tcolorboxenvironment{lemma}{
  boxrule=0pt,
  boxsep=2pt,
  blanker,
  borderline west={2pt}{0pt}{Red},
  before skip=10pt,
  after skip=10pt,
  sharp corners,
  left=12pt,
  right=12pt,
  breakable,
}

\tcolorboxenvironment{proof}{
  boxrule=0pt,
  boxsep=2pt,
  blanker,
  borderline west={2pt}{0pt}{NavyBlue!80!white},
  before skip=10pt,
  after skip=10pt,
  left=12pt,
  right=12pt,
  breakable,
}

\tcolorboxenvironment{remark}{
  boxrule=0pt,
  boxsep=2pt,
  blanker,
  borderline west={2pt}{0pt}{Green},
  before skip=10pt,
  after skip=10pt,
  left=12pt,
  right=12pt,
  breakable,
}

\tcolorboxenvironment{example}{
  boxrule=0pt,
  boxsep=2pt,
  blanker,
  borderline west={2pt}{0pt}{Black},
  sharp corners,
  before skip=10pt,
  after skip=10pt,
  left=12pt,
  right=12pt,
  breakable,
}

% ----------------------------------------------------------------------
% By jv start
\tcolorboxenvironment{problem}{
  boxrule=0pt,
  boxsep=2pt,
  colback={White!90!Melon},
  enhanced jigsaw,
  borderline west={2pt}{0pt}{Melon},
  sharp corners,
  before skip=1pt,
  after skip=1pt,
  breakable,
}
\tcolorboxenvironment{answer}{
  boxrule=0pt,
  boxsep=1pt,
  colback={White!90!White},
  enhanced jigsaw,
  borderline west={0pt}{0pt}{White},
  sharp corners,
  before skip=1pt,
  after skip=1pt,
  breakable,
}
\tcolorboxenvironment{solution}{
  boxrule=0pt,
  boxsep=1pt,
  colback={White!90!White},
  enhanced jigsaw,
  borderline west={0pt}{0pt}{Black},
  sharp corners,
  before skip=5pt,
  after skip=5pt,
  breakable,
}
% By jv end
% ----------------------------------------------------------------------
