\usepackage[english]{babel}
\usepackage[utf8]{inputenc}
\usepackage[T1]{fontenc}
\usepackage{lmodern}
\usepackage{microtype}
\usepackage{natbib}
\usepackage{tocbibind}
\usepackage{amsmath}
\usepackage{amsthm}
\usepackage{blindtext}
% \usepackage[a5paper]{geometry}
\usepackage[a4paper, inner=1.5cm, outer=3cm, top=2.5cm, bottom=2.5cm, bindingoffset=1.5cm]{geometry}
%- For line spacing
\usepackage[onehalfspacing]{setspace}
%-- For math symbols
\usepackage{amsmath}

%-- This introduces hyper rereferences in document : The print version
%-- would not be affected—neither the bookmarks nor those red frames would be printed
\usepackage{hyperref}

%-- Add this
\usepackage{titlesec}

%-- for drawings
\usepackage{tikz}

%--- Preamble : here you set new rules
    %--- For foonote  (else you can use \protect\footnote{Your note here} )
\renewcommand{\footnoterule}{\noindent\smash{\rule[3pt]{\textwidth}{0.4pt}}}
    %--- You can use this to name the contents, real place is tableofcontents below
\renewcommand{\contentsname}{Table of Contents}

%--- Customizing headers: Start
\usepackage{fancyhdr}
\fancyhf{}
\fancyhead[LE]{\leftmark}
\fancyhead[RO]{\nouppercase{\rightmark}}
\fancyfoot[LE,RO]{\thepage}
    %Switches to the page style fancy from this point onwards
\pagestyle{fancy}
%--- Customizing headers: End


%-- to specify layout and font of the chapter headings
%-- titlesec, which provides a comprehensive interface for customizing headings, of parts,
%-- chapters, sections, and even smaller sectioning parts down
%-- to subparagraphs.

\titleformat{\chapter}[display]
 {\normalfont\sffamily\Large\bfseries\raggedright}
 {\chaptertitlename\ \thechapter}{0pt}{\Huge}

%-- define the section heading by calling \titleformat again
\titleformat{\section}
 {\normalfont\sffamily\large\bfseries\centering}
 {\thesection}{1em}{}

%--  to adjust the chapter headings spacing:
\titlespacing*{\chapter}{0pt}{30pt}{20pt}


\newtheorem{thm}{Theorem}[chapter]
\newtheorem{lem}[thm]{Lemma}
\theoremstyle{definition}
\newtheorem{dfn}[thm]{Definition}

%--- pdf medatadate:
\hypersetup{pdfauthor={Chimpoo},
 pdftitle={The Big Book of Equations},
 pdfsubject={Solving Equations and Equation Systems},
 pdfkeywords={equations,mathematics}}


%--- Those quotations
\usepackage{epigraph}
\setlength{\epigraphwidth}{0.35\linewidth}
\setlength{\epigraphrule}{0pt}
\renewcommand*{\textflush}{flushright}
\renewcommand*{\epigraphsize}{\normalsize\itshape}


%-----------------------------------------------------------------------------
%-----------------------------------------------------------------------------
%-----------------------------------------------------------------------------
%-----------------------------------------------------------------------------
%% NotesTeX.sty
%% A package modified from NotesTeX.sty from
%%        https://jhep.sissa.it/jhep/help/JHEP_TeXclass.jsp
%% by Aditya Dhumuntarao.
%% ------------------------------- Legal -------------------------------
%% Adity Dhumuntarao does not own the copyright to the original package,
%% jheppub.sty. All modification have been approved by the Jhep Editori-
%% al committee, and permission has been attributed to Aditya to distri-
%% bute freely the modified version of jheppub.sty, known as NotesTeX.sty
%%
%% This work may be distributed and/or modified under the conditions of
%% the LaTeX Project Public License, either version 1.3 of this license
%% or (at your option) any later version. The latest version of this
%% license is in
%%        http://www.latex-project.org/lppl.txt
%% and version 1.3 or later is part of all distributions of LaTeX version
%% 2005/12/01 or later.
%% The Current Maintainer of this work is
%%        Aditya Dhumuntarao <adhumunt@gmail.com>
%% --------------------------------------------------------------------


 \NeedsTeXFormat{LaTeX2e}
 \ProvidesPackage{NotesTeX}[2017/03/21 r1]

 \gdef\@fpheader{\ }
 \gdef\@journal{jhep}

 \newif\ifnotoc\notocfalse
 \newif\ifemailadd\emailaddfalse
 \newif\iftoccontinuous\toccontinuousfalse
 \newif\ifnatbibsort\natbibsorttrue

 \DeclareOption{no-natbib-sort}{\natbibsortfalse}
 \ProcessOptions\relax

 % ----------------------------------------------------------------------
 %           User Top Level Packages: Required
 % ----------------------------------------------------------------------

 \usepackage{marginnote,sidenotes,fancyhdr,titlesec,geometry}
 %--\usepackage[dvipsnames]{xcolor}
 \usepackage[many]{tcolorbox}


% ----------------------------------------------------------------------
%           User Top Level Packages: Additional & Styling
% ----------------------------------------------------------------------

\usepackage[T1]{fontenc}                            % Font Styling
\usepackage{lmodern,mathrsfs}

\usepackage[shortlabels]{enumitem}

\usepackage{mathtools,amssymb,amsfonts,amsthm,bm}   % Math Presets
\usepackage{array,tabularx,booktabs}                % Table Presets
\usepackage{graphicx,wrapfig,float,caption}         % Figure Presets
\usepackage{setspace,multicol}                      % Text Presets
\usepackage{tikz,physics,cancel}                    % Physics Presets


% ----------------------------------------------------------------------
%           User Page Prefrences
% ----------------------------------------------------------------------
\DeclareGraphicsExtensions{.pdf,.png,.jpg}

%\geometry{paperheight=845pt,paperwidth=597pt,                   %fix paperwidth and height.
%          marginparsep=.02\paperwidth,marginparwidth=.23\paperwidth,
%          inner=.05\paperwidth,voffset=-1in,headheight=.02\paperheight,
%          headsep=.03\paperheight,footskip=20pt,
%          textheight=.84\paperheight,textwidth=.64\paperwidth}



%-- \pagestyle{fancy}%
\newlength{\myoddoffset}%
\setlength{\myoddoffset}{\marginparwidth + \marginparsep}%
\renewcommand{\sectionmark}[1]{\markboth{#1}{}}%
\renewcommand{\subsectionmark}[1]{\markright{#1}{}}%



\fancypagestyle{fancynotes}{%
  \fancyhf{}%
  \fancyheadoffset[rh]{\myoddoffset}%
  \renewcommand{\headrulewidth}{0pt}%
  \fancyhead[L]{\textsc{\leftmark}}%
  \fancyhead[R]{\footnotesize \textit{\rightmark}~~~~ \thepage}%
}%




\fancypagestyle{fancypart}{%
  \fancyhf{}%
  \fancyfootoffset[rh]{\myoddoffset}%
  \renewcommand{\headrulewidth}{0pt}
  \fancyfoot[L]{\footnotesize \thepage}%
}%

\titleformat{\part}[hang]{{\thispagestyle{fancypart}}\Huge\bfseries}{\marginnote{
\begin{tcolorbox}[width=\marginparwidth,height=\marginparwidth/2,colback=black!75!white,colframe=black!75!white,center title,fonttitle=\bfseries\normalsize,title=PART,text fill]
  \begin{center}
  {\color{white}\thepart}
  \end{center}
\end{tcolorbox}
}[-1.25in]}{0pt}{\sffamily\Huge\bfseries}

\newenvironment{fullpage}
    {\smallskip\noindent\begin{minipage}
    {\textwidth+\marginparwidth+\marginparsep}\hrule\smallskip\smallskip}
    {\hrule\end{minipage}\vspace{.1in}}



    % ----------------------------------------------------------------------
    %           User Created Environments
    % ----------------------------------------------------------------------


    \newtheoremstyle{mystyle}%
      {}%
      {}%
      {}%
      {}%
      {\sffamily\bfseries}%
      {.}%
      { }%
      {}%

    \renewenvironment{proof}{{\sffamily\bfseries Proof. }}{\qed}

    \theoremstyle{mystyle}{
      \newtheorem*{remark}{Remark}
    }

    \theoremstyle{mystyle}{
      \newtheorem{definition}{Definition}[section]
    }

    \theoremstyle{mystyle}{
      \newtheorem{theorem}{Theorem}[section]
    }

    \theoremstyle{mystyle}{
      \newtheorem{lemma}[theorem]{Lemma}
    }

    \theoremstyle{mystyle}{
      \newtheorem*{example}{Example}
    }

    \theoremstyle{definition}{
        \newtheorem*{exercise}{Exercise}
    }

    % ----------------------------------------------------------------------
    % start: By jv
    \theoremstyle{mystyle}{
      \newtheorem{problem}{Problem}[section]
    }
    \theoremstyle{mystyle}{
      \newtheorem{answer}{Answers}[section]
    }
    \theoremstyle{mystyle}{
      \newtheorem{solution}{Solution}[section]
    }
    % end: By jv
    % ----------------------------------------------------------------------



    \tcolorboxenvironment{definition}{
      boxrule=0pt,
      boxsep=2pt,
      colback={White!90!Cerulean},
      enhanced jigsaw,
      borderline west={2pt}{0pt}{Cerulean},
      sharp corners,
      before skip=10pt,
      after skip=10pt,
      breakable,
    }

    \tcolorboxenvironment{theorem}{
      boxrule=0pt,
      boxsep=2pt,
      colback={White!90!Dandelion},
      enhanced jigsaw,
      borderline west={2pt}{0pt}{Dandelion},
      sharp corners,
      before skip=10pt,
      after skip=10pt,
      breakable,
    }

    \tcolorboxenvironment{lemma}{
      boxrule=0pt,
      boxsep=2pt,
      blanker,
      borderline west={2pt}{0pt}{Red},
      before skip=10pt,
      after skip=10pt,
      sharp corners,
      left=12pt,
      right=12pt,
      breakable,
    }

    \tcolorboxenvironment{proof}{
      boxrule=0pt,
      boxsep=2pt,
      blanker,
      borderline west={2pt}{0pt}{NavyBlue!80!white},
      before skip=10pt,
      after skip=10pt,
      left=12pt,
      right=12pt,
      breakable,
    }

    \tcolorboxenvironment{remark}{
      boxrule=0pt,
      boxsep=2pt,
      blanker,
      borderline west={2pt}{0pt}{Green},
      before skip=10pt,
      after skip=10pt,
      left=12pt,
      right=12pt,
      breakable,
    }

    \tcolorboxenvironment{example}{
      boxrule=0pt,
      boxsep=2pt,
      blanker,
      borderline west={2pt}{0pt}{Black},
      sharp corners,
      before skip=10pt,
      after skip=10pt,
      left=12pt,
      right=12pt,
      breakable,
    }

    % ----------------------------------------------------------------------
    % By jv start
    \tcolorboxenvironment{problem}{
      boxrule=0pt,
      boxsep=2pt,
      colback={White!90!Melon},
      enhanced jigsaw,
      borderline west={2pt}{0pt}{Melon},
      sharp corners,
      before skip=10pt,
      after skip=10pt,
      breakable,
    }
    \tcolorboxenvironment{answer}{
      boxrule=0pt,
      boxsep=0pt,
      colback={White!90!White},
      enhanced jigsaw,
      borderline west={1pt}{0pt}{White},
      sharp corners,
      before skip=10pt,
      after skip=10pt,
      breakable,
    }
    \tcolorboxenvironment{solution}{
      boxrule=0pt,
      boxsep=2pt,
      colback={White!90!White},
      enhanced jigsaw,
      borderline west={1pt}{0pt}{Black},
      sharp corners,
      before skip=10pt,
      after skip=10pt,
      breakable,
    }
    % By jv end
    % ----------------------------------------------------------------------



    \renewcommand*{\raggedleftmarginnote}{\noindent}
    \renewcommand*{\raggedrightmarginnote}{\noindent}
    \newcommand{\mn}[1]{\textsuperscript{\thesidenote}{}\marginnote{\textsuperscript{\thesidenote}{}~\itshape\footnotesize #1}\refstepcounter{sidenote}}
    \newcommand{\en}[1]{\marginnote{\footnotesize #1}}
    \newcommand{\sn}[1]{\sidenote{\itshape\footnotesize #1}}

    % ----------------------------------------------------------------------
    %           User Created Commands
    % ----------------------------------------------------------------------

    \newcommand*\widefbox[1]{\fbox{\hspace{2em}#1\hspace{2em}}}
    \newcommand{\xint}{\int_{x_1}^{x_2}}
    \newcommand{\tint}{\int_{t_1}^{t_2}}
    \newcommand{\mw}{\sqrt{m\omega}}
    \newcommand{\de}{\delta}
    \newcommand{\dde}{\dot{\delta}}
    \newcommand{\di}{\delta_i}
    \newcommand{\ddi}{\dot{\delta_i}}
    \newcommand{\dddi}{\ddot{\delta_i}}
    \newcommand{\dipl}{\delta_{i+1}}
    \newcommand{\dimi}{\delta_{i-1}}
    \newcommand{\ddt}[1]{\frac{{d} #1}{dt}}
    \newcommand{\ddtt}[1]{\frac{d^2 #1}{dt^2}}
    \newcommand{\ddx}[1]{\frac{d #1}{dx}}
    \newcommand{\ddxx}[1]{\frac{d^2 #1}{dx^2}}
    \newcommand{\eps}{\epsilon}
    \newcommand{\del}[2]{\frac{\partial #1}{\partial #2}}
    \newcommand{\deltwo}[2]{\frac{\partial^2 #1}{\partial #2^2}}
    \newcommand{\lam}{\lambda}
    \newcommand{\Lam}{\Lambda}
    \newcommand{\sig}{\sigma}
    \newcommand{\Sig}{\Sigma}
    \newcommand{\half}{\frac{1}{2}}
    \newcommand{\munu}{{\mu\nu}}
    \newcommand{\thalf}{\tfrac{1}{2}}

    \newcommand{\bfA}{{\bf A}}
    \newcommand{\bfB}{{\bf B}}
    \newcommand{\bfC}{{\bf C}}
    \newcommand{\bfD}{{\bf D}}
    \newcommand{\bfE}{{\bf E}}
    \newcommand{\bfF}{{\bf F}}
    \newcommand{\bfG}{{\bf G}}
    \newcommand{\bfH}{{\bf H}}
    \newcommand{\bfI}{{\bf I}}
    \newcommand{\bfJ}{{\bf J}}
    \newcommand{\bfK}{{\bf K}}
    \newcommand{\bfL}{{\bf L}}
    \newcommand{\bfM}{{\bf M}}
    \newcommand{\bfN}{{\bf N}}
    \newcommand{\bfO}{{\bf O}}
    \newcommand{\bfP}{{\bf P}}
    \newcommand{\bfQ}{{\bf Q}}
    \newcommand{\bfR}{{\bf R}}
    \newcommand{\bfS}{{\bf S}}
    \newcommand{\bfT}{{\bf T}}
    \newcommand{\bfU}{{\bf U}}
    \newcommand{\bfV}{{\bf V}}
    \newcommand{\bfW}{{\bf W}}
    \newcommand{\bfX}{{\bf X}}
    \newcommand{\bfY}{{\bf Y}}
    \newcommand{\bfZ}{{\bf Z}}

    \newcommand{\bfa}{{\bf a}}
    \newcommand{\bfb}{{\bf b}}
    \newcommand{\bfc}{{\bf c}}
    \newcommand{\bfd}{{\bf d}}
    \newcommand{\bfe}{{\bf e}}
    \newcommand{\bff}{{\bf f}}
    \newcommand{\bfg}{{\bf g}}
    \newcommand{\bfh}{{\bf h}}
    \newcommand{\bfi}{{\bf i}}
    \newcommand{\bfj}{{\bf j}}
    \newcommand{\bfk}{{\bf k}}
    \newcommand{\bfl}{{\bf l}}
    \newcommand{\bfm}{{\bf m}}
    \newcommand{\bfn}{{\bf n}}
    \newcommand{\bfo}{{\bf o}}
    \newcommand{\bfp}{{\bf p}}
    \newcommand{\bfq}{{\bf q}}
    \newcommand{\bfr}{{\bf r}}
    \newcommand{\bfs}{{\bf s}}
    \newcommand{\bft}{{\bf t}}
    \newcommand{\bfu}{{\bf u}}
    \newcommand{\bfv}{{\bf v}}
    \newcommand{\bfw}{{\bf w}}
    \newcommand{\bfx}{{\bf x}}
    \newcommand{\bfy}{{\bf y}}
    \newcommand{\bfz}{{\bf z}}

    \newcommand{\mcA}{{\mathcal{A}}}
    \newcommand{\mcB}{{\mathcal{B}}}
    \newcommand{\mcC}{{\mathcal{C}}}
    \newcommand{\mcD}{{\mathcal{D}}}
    \newcommand{\mcE}{{\mathcal{E}}}
    \newcommand{\mcF}{{\mathcal{F}}}
    \newcommand{\mcG}{{\mathcal{G}}}
    \newcommand{\mcH}{{\mathcal{H}}}
    \newcommand{\mcI}{{\mathcal{I}}}
    \newcommand{\mcJ}{{\mathcal{J}}}
    \newcommand{\mcK}{{\mathcal{K}}}
    \newcommand{\mcL}{{\mathcal{L}}}
    \newcommand{\mcM}{{\mathcal{M}}}
    \newcommand{\mcN}{{\mathcal{N}}}
    \newcommand{\mcO}{{\mathcal{O}}}
    \newcommand{\mcP}{{\mathcal{P}}}
    \newcommand{\mcQ}{{\mathcal{Q}}}
    \newcommand{\mcR}{{\mathcal{R}}}
    \newcommand{\mcS}{{\mathcal{S}}}
    \newcommand{\mcT}{{\mathcal{T}}}
    \newcommand{\mcU}{{\mathcal{U}}}
    \newcommand{\mcV}{{\mathcal{V}}}
    \newcommand{\mcW}{{\mathcal{W}}}
    \newcommand{\mcX}{{\mathcal{X}}}
    \newcommand{\mcY}{{\mathcal{Y}}}
    \newcommand{\mcZ}{{\mathcal{Z}}}

    \newcommand{\bbA}{{\mathbb{A}}}
    \newcommand{\bbB}{{\mathbb{B}}}
    \newcommand{\bbC}{{\mathbb{C}}}
    \newcommand{\bbD}{{\mathbb{D}}}
    \newcommand{\bbE}{{\mathbb{E}}}
    \newcommand{\bbF}{{\mathbb{F}}}
    \newcommand{\bbG}{{\mathbb{G}}}
    \newcommand{\bbH}{{\mathbb{H}}}
    \newcommand{\bbI}{{\mathbb{I}}}
    \newcommand{\bbJ}{{\mathbb{J}}}
    \newcommand{\bbK}{{\mathbb{K}}}
    \newcommand{\bbL}{{\mathbb{L}}}
    \newcommand{\bbM}{{\mathbb{M}}}
    \newcommand{\bbN}{{\mathbb{N}}}
    \newcommand{\bbO}{{\mathbb{O}}}
    \newcommand{\bbP}{{\mathbb{P}}}
    \newcommand{\bbQ}{{\mathbb{Q}}}
    \newcommand{\bbR}{{\mathbb{R}}}
    \newcommand{\bbS}{{\mathbb{S}}}
    \newcommand{\bbT}{{\mathbb{T}}}
    \newcommand{\bbU}{{\mathbb{U}}}
    \newcommand{\bbV}{{\mathbb{V}}}
    \newcommand{\bbW}{{\mathbb{W}}}
    \newcommand{\bbX}{{\mathbb{X}}}
    \newcommand{\bbY}{{\mathbb{Y}}}
    \newcommand{\bbZ}{{\mathbb{Z}}}

    \newcommand{\mfa}{{\mathfrak{a}}}
    \newcommand{\mfb}{{\mathfrak{b}}}
    \newcommand{\mfc}{{\mathfrak{c}}}
    \newcommand{\mfd}{{\mathfrak{d}}}
    \newcommand{\mfe}{{\mathfrak{e}}}
    \newcommand{\mff}{{\mathfrak{f}}}
    \newcommand{\mfg}{{\mathfrak{g}}}
    \newcommand{\mfh}{{\mathfrak{h}}}
    \newcommand{\mfi}{{\mathfrak{i}}}
    \newcommand{\mfj}{{\mathfrak{j}}}
    \newcommand{\mfk}{{\mathfrak{k}}}
    \newcommand{\mfl}{{\mathfrak{l}}}
    \newcommand{\mfm}{{\mathfrak{m}}}
    \newcommand{\mfn}{{\mathfrak{n}}}
    \newcommand{\mfo}{{\mathfrak{o}}}
    \newcommand{\mfp}{{\mathfrak{p}}}
    \newcommand{\mfq}{{\mathfrak{q}}}
    \newcommand{\mfr}{{\mathfrak{r}}}
    \newcommand{\mfs}{{\mathfrak{s}}}
    \newcommand{\mft}{{\mathfrak{t}}}
    \newcommand{\mfu}{{\mathfrak{u}}}
    \newcommand{\mfv}{{\mathfrak{v}}}
    \newcommand{\mfw}{{\mathfrak{w}}}
    \newcommand{\mfx}{{\mathfrak{x}}}
    \newcommand{\mfy}{{\mathfrak{y}}}
    \newcommand{\mfz}{{\mathfrak{z}}}

    \newcommand{\mfA}{{\mathfrak{A}}}
    \newcommand{\mfB}{{\mathfrak{B}}}
    \newcommand{\mfC}{{\mathfrak{C}}}
    \newcommand{\mfD}{{\mathfrak{D}}}
    \newcommand{\mfE}{{\mathfrak{E}}}
    \newcommand{\mfF}{{\mathfrak{F}}}
    \newcommand{\mfG}{{\mathfrak{G}}}
    \newcommand{\mfH}{{\mathfrak{H}}}
    \newcommand{\mfI}{{\mathfrak{I}}}
    \newcommand{\mfJ}{{\mathfrak{J}}}
    \newcommand{\mfK}{{\mathfrak{K}}}
    \newcommand{\mfL}{{\mathfrak{L}}}
    \newcommand{\mfM}{{\mathfrak{M}}}
    \newcommand{\mfN}{{\mathfrak{N}}}
    \newcommand{\mfO}{{\mathfrak{O}}}
    \newcommand{\mfP}{{\mathfrak{P}}}
    \newcommand{\mfQ}{{\mathfrak{Q}}}
    \newcommand{\mfR}{{\mathfrak{R}}}
    \newcommand{\mfS}{{\mathfrak{S}}}
    \newcommand{\mfT}{{\mathfrak{T}}}
    \newcommand{\mfU}{{\mathfrak{U}}}
    \newcommand{\mfV}{{\mathfrak{V}}}
    \newcommand{\mfW}{{\mathfrak{W}}}
    \newcommand{\mfX}{{\mathfrak{X}}}
    \newcommand{\mfY}{{\mathfrak{Y}}}
    \newcommand{\mfZ}{{\mathfrak{Z}}}
