\documentclass{article}

\usepackage[utf8]{inputenc}
\usepackage{physics}
\usepackage{pict2e}
\usepackage{scalerel}
\usepackage{tkz-euclide, tkz-base}
\usepackage{tikz}
\usepackage{pgfplots}
\pgfplotsset{compat=newest}
\usetikzlibrary{calc, patterns,angles,quotes, arrows.meta, shadows, external}
\usepgfplotslibrary{statistics}
\usepgfplotslibrary{fillbetween}

\title{Plotting Ticz}
\author{JV}
\begin{document}
\maketitle


This is a test paragraph.

\begin{tikzpicture}
    \draw[->] (-0.1, 0) -- (5.0,0);
    \draw[->] (0, -0.1) -- (0,5.0); 
    \draw[gray, ultra thin] (-0.1, -0.1) grid (5.1,5.1); 
    \draw[domain=0:2, orange, thick] plot (\x, {exp(\x/2)});
    \draw[domain=0:2, blue, thick] plot (\x, {\x*\x});
\end{tikzpicture}

We define domain on top near tikzpicture below:

\begin{tikzpicture}[domain = 0:2]
    \draw[->] (-0.1, 0) -- (5.0,0);
    \draw[->] (0, -0.1) -- (0,5.0); 
    \draw[gray, ultra thin] (-0.1, -0.1) grid (5.1,5.1); 
    \draw[ orange, thick] plot (\x, {exp(\x/2)});
    \draw[ blue, thick] plot (\x, {\x*\x});
\end{tikzpicture}

\pagebreak

Label like this below:

\begin{tikzpicture}[domain = 0:2]
    \draw[->] (-0.1, 0) -- (5.0,0);
    \draw[->] (0, -0.1) -- (0,5.0); 
    \draw[gray, ultra thin] (-0.1, -0.1) grid (5.1,5.1);
    \draw[ orange, thick] plot (\x, {exp(\x/2)}) node[right]{$f(x) = e^{x/2}$};
    \draw[ blue, thick] plot (\x, {\x*\x}) node[above] {$g(x) = {x^2}$};
\end{tikzpicture}

Center this picture:

\begin{center}
    \begin{tikzpicture}[domain = 0:2]
        \draw[->] (-0.1, 0) -- (5.0,0);
        \draw[->] (0, -0.1) -- (0,5.0); 
        \draw[gray, ultra thin] (-0.1, -0.1) grid (5.1,5.1);
        \draw[ orange, thick] plot (\x, {exp(\x/2)}) node[right]{$f(x) = e^{x/2}$};
        \draw[ blue, thick] plot (\x, {\x*\x}) node[above] {$g(x) = {x^2}$};
    \end{tikzpicture}    
\end{center}

See the coordinate system/addition:

\begin{center}
    \begin{tikzpicture}[domain = 0:2]
        \coordinate (A) at (-1, 0, 0);
        \coordinate (B) at ( 1, 0, 0);
        \coordinate (C) at (0, 0, {sqrt(3)});
        \coordinate (H) at (0, 2, 0);
        \draw (B) -- (C) -- (A) -- cycle;
        \draw ($(B)+(H)$) -- ($(C)+(H)$) -- ($(A)+(H)$) -- cycle;
        \draw[dashed] (A)  -- ($(A) + (H)$);
        \draw[dashed] (B)  -- ($(B) + (H)$);
        \draw[dashed] (C)  -- ($(C) + (H)$);
    \end{tikzpicture}
    \end{center}

\pagebreak
Integral thing (dashed area):

\begin{center}
    \begin{tikzpicture}[domain = 0:2]
        \draw[->, thick] (-1, 0) -- (3.0, 0);
        \draw[->, thick] (0, -1) -- (0, 3.0); 
        %\draw[gray, ultra thin] (-1, -1) grid (3,5);
        \draw[ blue, thick] plot (\x, {exp(\x/2)}) node[right]{$f(x) = e^{x/2}$};
        \draw[dashed] (1, 0) -- (1,  {exp(1/2)});
        \draw[dashed] (2, 0) -- (2,  {exp(2/2)});

    \end{tikzpicture}    
\end{center}

Integral with shaded area:


\begin{tikzpicture}[domain = 0:2]
    \begin{axis}[axis lines=middle,
        xlabel=$x$,
        ylabel=$y$,
        enlargelimits,
        ytick=\empty,
        xtick={1,4},
        xticklabels={a,b}]
    \addplot[name path=F,blue,domain={-.2:5}] {exp(\x/2)} node[pos=0.8, left]{$f(x) = e^{\frac{x}{2}}$};
    \addplot[name path=G,gray, thin,domain={0:5}] {0};
    \addplot[pattern=north west lines, pattern color=brown!50]fill between[of=F and G, soft clip={domain=1:4}];
    \end{axis}
\end{tikzpicture}    

\pagebreak

Integral with shaded area, with figure:

\begin{figure}[h]
\begin{tikzpicture}[domain = 0:2]
    \begin{axis}[axis lines=middle,
        xlabel=$x$,
        ylabel=$y$,
        enlargelimits,
        ytick=\empty,
        xtick={1,4},
        xticklabels={a,b}]
    \addplot[name path=F,blue,domain={-.2:5}] {exp(\x/2)} node[pos=0.8, left]{$f(x) = e^{\frac{x}{2}}$};
    \addplot[name path=G,gray, thin,domain={0:5}] {0};
    \addplot[pattern=north west lines, pattern color=brown!50]fill between[of=F and G, soft clip={domain=1:4}];
    \end{axis}
\end{tikzpicture}
\caption{This is Area under the curve}
\end{figure}

Integral with shaded area, with figure and centering:

\begin{figure}[h]
\begin{tikzpicture}[domain = 0:2]
    \begin{axis}[axis lines=middle,
        xlabel=$x$,
        ylabel=$y$,
        enlargelimits,
        ytick=\empty,
        xtick={1,4},
        xticklabels={a,b}]
    \addplot[name path=F,blue,domain={-.2:5}] {exp(\x/2)} node[pos=0.8, left]{$f(x) = e^{\frac{x}{2}}$};
    \addplot[name path=G,gray, thin,domain={0:5}] {0};
    \addplot[pattern=north west lines, pattern color=brown!50]fill between[of=F and G, soft clip={domain=1:4}];
    \end{axis}
\end{tikzpicture}
\caption{This is Area under the curve}
\end{figure}

\pagebreak
With polynomial kind :

\begin{figure}[h]
\begin{tikzpicture}[domain = -5:5]
        \begin{axis}[axis lines=middle,
            xlabel=$x$,
            ylabel=$y $,
            enlargelimits,
            ytick=\empty,
            xtick={-3,1,4},
            xticklabels={-3,1,2}]
        \addplot[name path=F,blue,domain={-3.5:5}] {(\x+3)*(\x-1)*(\x-4)} node[pos=0.8, left]{$f(x) = (x+3)(x-1)(x-2)$};
        \end{axis}
    \end{tikzpicture}
    \caption{Cubic Polynomial}
\end{figure}

With figure trial:
    
\begin{figure}[h]
        \begin{tikzpicture}[domain = -5:5]
            \begin{axis}[axis lines=middle,
                xlabel=$x$,
                ylabel=$y $,
                enlargelimits,
                ytick=\empty,
                xtick={-3,1,4},
                xticklabels={-3,1,2}]
            \addplot[name path=F,blue,domain={-3.5:5}] {(\x+3)*(\x-1)*(\x-4)} node[pos=0.8, left]{\tiny $f(x) = (x+3)(x-1)(x-2)$};
            \end{axis}
        \end{tikzpicture}
        \caption{A cubic polynomial}
\end{figure}
        
\begin{figure}[h]
        \begin{tikzpicture}[domain = -5:5]
            \begin{axis}[axis lines=middle,
                xlabel=$x$,
                ylabel=$y $,
                enlargelimits,
                ytick=\empty,
                xtick={-3,1,4},
                xticklabels={-3,1,2}]
            \addplot[name path=F,blue,domain={-3.5:5}] {(\x+3)*(\x-1)*(\x-4)^2} node[pos=0, above]{\tiny $f(x)$};
            \end{axis}
        \end{tikzpicture}
        \caption{$f(x) = (x+3)(x-1)(x-2)^2$}
\end{figure}

\pagebreak

Circle

\begin{center}
    \begin{tikzpicture}[domain = -5:5]
        \def\cy{3.0};
        \def\cx{0.0};
        \coordinate (C) at (\cx,\cy);
        \def\r{2.0};
        \def\ang{30};
        \def\q{\cy-0};
        \def\x{{\r^2/\q}}; % Q x coordinate
        \def\y{{\r*sqrt(1-(\r/\q)^2}}; % Q y coordinate
        \coordinate (A) at (\x,\y); % point of tangency, A
        \coordinate (Q) at (0,0); % external point Q
        \coordinate (R) at ({\ang}:{\r});
        \coordinate (Z1) at ({120+\ang}:{\r});
        \coordinate (H) at (\r,0);
        \draw[->, thick] (-5, 0) -- (5, 0) node[right] {$Im(z)$};
        \draw[->, thick] (0, -1) -- (0, 6) node[left] {$Re(z)$};
        \draw (C) circle (\r);
        \fill[red] (C) -- ++(Z1)  circle(0.05) node[above] {$Z_1$};
        \fill[red] (A) circle(0.05) node[above] {A};
        \draw[orange] ($(Q)!-0.2!(A)$) -- ($(Q)!3!(A)$);
        \draw[-{Stealth[scale=2]}, dashed] (C) -- ++(R) node[midway, above left]{$r$};
    \end{tikzpicture}    
\end{center}

\pagebreak

Complex numbers marked on circle

\begin{center}
    \begin{tikzpicture}[domain = -5:5]
        %\def\cy{3.0};
        %\def\cx{0.0};
        \coordinate (O) at (0,0);
        \def \rout {4.0};
        \def \rin {2.0};
        \def \angout{30};
        \coordinate (Z1) at ({\angout}:{\rout});
        \coordinate (Z2) at ({15 + \angout}:{\rin});
        \coordinate (Z3) at ({180 + \angout}:{\rout});
        \coordinate (Z4) at ({270 + \angout}:{\rout});
        \draw[->, thick] (0, -5) -- (0, 5) node[left] {$Im(z)$};
        \draw[->, thick] (-5, 0) -- (5, 0) node[right] {$Re(z)$};
        \draw (O) circle (\rout);
        \draw (O) circle (\rin);
        \fill[red] (O) -- ++(Z1)  circle(0.05) node[above] {$Z_1$};
        \fill[red] (O) -- ++(Z2)  circle(0.05) node[above] {$Z_2$};
        \fill[red] (O) -- ++(Z3)  circle(0.05) node[above] {$Z_3$};
        \fill[red] (O) -- ++(Z4)  circle(0.05) node[above] {$Z_4$};
        \node[above left, outer sep=2pt] (O) {$O$};
        %\draw[->, orange] (0,0) -- (Z1);
        %\draw[-{Stealth[scale=2]}, dashed] (C) -- ++(R) node[midway, above left]{$r$};
    \end{tikzpicture}    
\end{center}

\end{document}